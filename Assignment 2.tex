\documentclass[journal,12pt,twocolumn]{IEEEtran}

\usepackage{setspace}
\usepackage{gensymb}

\singlespacing


\usepackage[cmex10]{amsmath}

\usepackage{amsthm}

\usepackage{mathrsfs}
\usepackage{txfonts}
\usepackage{stfloats}
\usepackage{bm}
\usepackage{cite}
\usepackage{cases}
\usepackage{subfig}


\usepackage{longtable}
\usepackage{multirow}

\usepackage{enumitem}
\usepackage{mathtools}
\usepackage{steinmetz}
\usepackage{tikz}
\usepackage{circuitikz}
\usepackage{verbatim}
\usepackage{tfrupee}
\usepackage[breaklinks=true]{hyperref}
\raggedbottom

\usepackage{tkz-euclide}

\usetikzlibrary{calc,math}
\usepackage{listings}
    \usepackage{color}                                            %%
    \usepackage{array}                                            %%
    \usepackage{longtable}                                        %%
    \usepackage{calc}                                             %%
    \usepackage{multirow}                                         %%
    \usepackage{hhline}                                           %%
    \usepackage{ifthen}                                           %%
    \usepackage{lscape}     
\usepackage{multicol}
\usepackage{chngcntr}
\graphicspath{ {./images/} }

\DeclareMathOperator*{\Res}{Res}
\newcommand*{\prob}{\mathsf{P}}

\renewcommand\thesection{\arabic{section}}
\renewcommand\thesubsection{\thesection.\arabic{subsection}}
\renewcommand\thesubsubsection{\thesubsection.\arabic{subsubsection}}

\renewcommand\thesectiondis{\arabic{section}}
\renewcommand\thesubsectiondis{\thesectiondis.\arabic{subsection}}
\renewcommand\thesubsubsectiondis{\thesubsectiondis.\arabic{subsubsection}}


\hyphenation{op-tical net-works semi-conduc-tor}
\def\inputGnumericTable{}                                 %%

\lstset{
%language=C,
frame=single, 
breaklines=true,
columns=fullflexible
}
\begin{document}

\newtheorem{theorem}{Theorem}[section]
%\def\@opargbegintheorem#1#2#3{\trivlist\item[]{\bfseries #1\ #2\ (#3)} \itshape}
\newtheorem{problem}{Problem}
\newtheorem{proposition}{Proposition}[section]
\newtheorem{lemma}{Lemma}[section]
%\newtheorem{corollary}[theorem]{Corollary}
\newtheorem{example}{Example}[section]
\newtheorem{definition}[problem]{Definition}

\newcommand{\BEQA}{\begin{eqnarray}}
\newcommand{\EEQA}{\end{eqnarray}}
\newcommand{\define}{\stackrel{\triangle}{=}}
\bibliographystyle{IEEEtran}
\providecommand{\mbf}{\mathbf}
\providecommand{\pr}[1]{\ensuremath{\Pr\left(#1\right)}}
\providecommand{\qfunc}[1]{\ensuremath{Q\left(#1\right)}}
\providecommand{\sbrak}[1]{\ensuremath{{}\left[#1\right]}}
\providecommand{\lsbrak}[1]{\ensuremath{{}\left[#1\right.}}
\providecommand{\rsbrak}[1]{\ensuremath{{}\left.#1\right]}}
\providecommand{\brak}[1]{\ensuremath{\left(#1\right)}}
\providecommand{\lbrak}[1]{\ensuremath{\left(#1\right.}}
\providecommand{\rbrak}[1]{\ensuremath{\left.#1\right)}}
\providecommand{\cbrak}[1]{\ensuremath{\left\{#1\right\}}}
\providecommand{\lcbrak}[1]{\ensuremath{\left\{#1\right.}}
\providecommand{\rcbrak}[1]{\ensuremath{\left.#1\right\}}}
\theoremstyle{remark}
\newtheorem{rem}{Remark}
\newcommand{\sgn}{\mathop{\mathrm{sgn}}}
\providecommand{\abs}[1]{\left\vert#1\right\vert}
\providecommand{\res}[1]{\Res\displaylimits_{#1}} 
\providecommand{\norm}[1]{\left\lVert#1\right\rVert}
%\providecommand{\norm}[1]{\lVert#1\rVert}
\providecommand{\mtx}[1]{\mathbf{#1}}
\providecommand{\mean}[1]{E\left[ #1 \right]}
\providecommand{\fourier}{\overset{\mathcal{F}}{ \rightleftharpoons}}
%\providecommand{\hilbert}{\overset{\mathcal{H}}{ \rightleftharpoons}}
\providecommand{\system}{\overset{\mathcal{H}}{ \longleftrightarrow}}
	%\newcommand{\solution}[2]{\textbf{Solution:}{#1}}
\newcommand{\solution}{\noindent \textbf{Solution: }}
\newcommand{\cosec}{\,\text{cosec}\,}
\providecommand{\dec}[2]{\ensuremath{\overset{#1}{\underset{#2}{\gtrless}}}}
\newcommand{\myvec}[1]{\ensuremath{\begin{pmatrix}#1\end{pmatrix}}}
\newcommand{\mydet}[1]{\ensuremath{\begin{vmatrix}#1\end{vmatrix}}}
\numberwithin{equation}{subsection}
\makeatletter
\@addtoreset{figure}{problem}
\makeatother
\let\StandardTheFigure\thefigure
\let\vec\mathbf
\renewcommand{\thefigure}{\theproblem}
\def\putbox#1#2#3{\makebox[0in][l]{\makebox[#1][l]{}\raisebox{\baselineskip}[0in][0in]{\raisebox{#2}[0in][0in]{#3}}}}
     \def\rightbox#1{\makebox[0in][r]{#1}}
     \def\centbox#1{\makebox[0in]{#1}}
     \def\topbox#1{\raisebox{-\baselineskip}[0in][0in]{#1}}
     \def\midbox#1{\raisebox{-0.5\baselineskip}[0in][0in]{#1}}
\vspace{3cm}
\title{Assignment 2}
\author{N PRAFUL RAJ(CC20MTECH11004)}
\maketitle
\newpage
\bigskip
\renewcommand{\thefigure}{\theenumi}
\renewcommand{\thetable}{\theenumi}

\section{Problem}
A man is known to speak truth 3 out of 4 times. He throws a die and reports that it is a six. Find the probability that it is actually a six.

\section{Explanation}
By reading the problem we can understand that the probability of an event is asked to findout based on the occurence of the other event.Here the Probability of occurence of six has been asked based on the occurence of the another event i.e reporting of the occurence of the six.Such kind of conditional probabilities can be solved using Bayes Theorem.
\begin{theorem}[Bayes Theorem]
 If $E_{1}$, $E_{2}$,..., $E_{n}$are n non empty events which constitute a partition of sample space S, i.e.$E_{1}$ , $E_{2}$ ,...,$E_{n}$  are pairwise disjoint and $E_{1} \bigcup E_{2} \bigcup ........E_{n-1} \bigcup E_{n}$= S and A is any event of nonzero probability, then \begin{align}\label{bayes}
\prob(E_{i}|A) = \dfrac{\prob(E_{i}) \prob(A|E_{i})}{ \sum_{j=1}^{n} \prob(E_{j}) \prob(A|E_{j})}  
 \end{align}
 for any i=1,2,3,.....,n
 \end{theorem}
 
\section{Solution}
Let $E_{1}$ be the event where six occurs
\begin{align}\label{1}
\prob(E_{1})=\dfrac{1}{6}
\end{align}
Let $E_{2}$ be the event where six doesnot occur
\begin{align}\label{2}
\prob(E_{2})=\dfrac{5}{6}
\end{align}
Let A be an event where the man reports the occurence of six,then the probability that man reports occurence of six only when six has occured \begin{align}\label{3}
\prob({A|E_{1}})=\dfrac{3}{4}
\end{align}

Probability that the man reports occurence of six  when six has not occured \begin{align}\label{4}
\prob({A|E_{2}})=\dfrac{1}{4}
\end{align}

$\therefore$ The probability that man reports six while he is telling truth using \eqref{bayes} is 
\begin{align}\label{anseq}
\prob(E_{1}|A)=\dfrac{\prob(E_{1}) \prob(A|E_{1})}{\prob(E_{1}) \prob(A|E_{1})+\prob(E_{2}) \prob(A|E_{2})}
\end{align}
 Using \eqref{1},\eqref{2},\eqref{3},\eqref{4} in \eqref{anseq} 
 \begin{align}\label{ans}
 \dfrac{\dfrac{1}{6}*\dfrac{3}{4}}{\dfrac{1}{6}*\dfrac{3}{4}+\dfrac{5}{6}*\dfrac{1}{4}}=\dfrac{3}{8}
 \end{align}
 
  \begin{align}
  \prob(E_{1}|A)=\dfrac{3}{8}
  \end {align}

 
 




\end{document}